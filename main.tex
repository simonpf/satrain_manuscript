\documentclass[11pt]{article}
\usepackage{amsmath, amssymb, graphicx}
\title{A Benchmark Dataset for Satellite-Based Estimation and Detection of Rain}
\author{%
Simon Pfreundschuh\textsuperscript{1}, %
Malarvizhi Arulraj\textsuperscript{2}, %
Ali Behrangi\textsuperscript{3}, %
Linda Bogerd\textsuperscript{3}, \\
Alan James Peixoto Calheiros\textsuperscript{3}, %
Daniele Casella\textsuperscript{3}, %
Neda Dolatabadi\textsuperscript{3}, \\ %
Clement Guilloteau\textsuperscript{2}, %
Jie Gong\textsuperscript{3}, %
Pierre Kirstetter\textsuperscript{3}, %
GyuWon Lee\textsuperscript{3}, \\ %
Maximilian Maahn\textsuperscript{3}, %
Lisa Milani\textsuperscript{3}, %
Rayana Palharini\textsuperscript{3},\\%
Veljko Petković\textsuperscript{3}, %
Soorok Ryu\textsuperscript{3}, %
Paolo Sanò\textsuperscript{3}, %
Jackson Tan\textsuperscript{3}\\
}
\date{
  \textsuperscript{2} University of California Irvine, Department of Civil and Environmental Engineering
  \textsuperscript{3} Department of Hydrology and Atmospheric Sciences, University of Arizona\\
  \textsuperscript{2} Department of Mathematics, University B, City, Country \\
  \textsuperscript{3} Department of Atmospheric Science, Colorado State University \\
  \textsuperscript{4} Instituto Nacional de Pesquisas Espaciais \\
  \textsuperscript{5} Department of Atmospheric Sciences, Kyungpook National University \\
  \textsuperscript{5} Department of Atmospheric Sciences, Kyungpook National University \\
  \textsuperscript{6} NASA Goddard Space Flight Center \\
  \textsuperscript{7} University of Maryland, Baltimore County \\
  \textsuperscript{7} University of Maryland\\
  \textsuperscript{8} Institute for Meteorology, University of Leipzig \\
  \textsuperscript{9} Institute of Atmospheric Sciences and Climate, Italian National Research Council \\
  \textsuperscript{10} Earth System Science Interdisciplinary Center, University of Maryland  \\
  \textsuperscript{10} Cooperative Institute for Satellite Earth System Studies, University of Maryland  \\
}

\date{}
\begin{document}
\maketitle

\begin{abstract}

Accurately monitoring the global distribution and evolution of precipitation is
essential for understanding precipitation processes and improving forecasts.
Satellite observations remain the only means of achieving consistent,
global-scale precipitation monitoring. While machine learning has long been
applied to satellite-based precipitation retrieval, the absence of a
standardized benchmark dataset has hindered fair comparisons between methods and
limited progress in algorithm development.

To address this gap, the International Precipitation Working Group has developed
SatRain, the first AI-ready benchmark dataset for satellite-based precipitation
detection and estimation. SatRain includes multi-sensor satellite observations
representative of the major platforms currently used in precipitation remote
sensing, paired with high-quality reference estimates from ground-based radars
corrected using rain gauge measurements. It offers a standardized evaluation
protocol to enable robust and reproducible comparisons across machine learning
approaches.

In addition to supporting algorithm evaluation, the diversity of sensors and
inclusion of time-resolved geostationary observations make SatRain a valuable
foundation for developing next-generation AI models to deliver more accurate,
detailed, and globally consistent precipitation estimates.

\end{abstract}

\section{Background and Summary}

Precipitation, the deposition of water in liquid or frozen form from the
atmosphere onto the Earth's surface, is essential for sustaining ecosystems and
a wide range of human activity. However, extreme events at both ends of the
climatological distribution of precipitation, such as droughts or heavy
precipitation, can cause substantial damage to societies and human livelihoods.
Monitoring the global distribution of precipitation is therefore critical not
only for advancing scientific understanding of the processes that shape
precipitation patterns and drive extreme events but also economic planning and
civil security. Despite its crucial role in many aspects of economic and social
life on Earth, precipitation estimation still faces significant challenges in
meeting the needs of hydrological and climate research, as well as operational
applications. Precipitation is one of the most difficult atmospheric parameters
to measure accurately, since its estimation from both satellite and ground-based
observations is complicated by several factors: its high spatial and temporal
variability; its phase (liquid, solid, or mixed); its microphysical compositions
(e.g., hydrometeor shape, densities, and sizes); and the difficulties involved
in converting radiometric measurements into quantitative precipitation
estimates.

Satellite-based remote sensing is an essential tool for monitoring precipitation
continuously on a global scale. Rain gauges provide valuable direct
measurements, but their coverage is largely confined to continental land masses
and is often irregularly distributed [1]. In addition, because gauges measure
precipitation only at a single point, they cannot adequately represent the
spatial structure of precipitation events. Ground-based weather radars offer
high-resolution, indirect estimates over limited regions, yet their geographic
coverage is insufficient for global precipitation monitoring. By contrast,
satellite observations provide consistent coverage worldwide, making them the
only measurement technique capable of delivering globally continuous and
consistent estimates of precipitation.

However, the accuracy with which precipitation can be estimated or detected from
satellite observations varies significantly with sensor type and observing
conditions [2]. While a small number of active sensors, i.e., radars, have been
deployed to measure precipitation from space, their spatial and temporal
coverages are limited, so global precipitation monitoring has to rely largely on
passive sensors. Passive microwave sensors operating in the 10-89 GHz range can
detect emission signals associated with precipitation particles over the ocean.
However, over land the emission signal from precipitation is hardly
distinguishable from the high and variable emission signal from the surface.
While higher microwave frequencies (> 89 GHz) are less sensitive to surface
properties, their information content primarily derives from scattering by large
rain drops and frozenice hydrometeorsparticles, providing a weaker link to the
precipitation at the surface. Furthermore, the spatial resolution achievable at
microwave frequencies is limited, requiring deployment in low-Earth orbits. As a
result, even for satellite missions comprising constellations of multiple
sensors, such as the Global Precipitation Measurement (GPM) mission [3], the
revisit times hardlycan’t exceed three hours in the tropics (30 °S - 30 °N.

In contrast, observations from geostationary platforms can provide temporally
continuous observations for much of the globe and achieve spatial resolutions of
the order of a few kilometers. The principal disadvantage of these sensors is
their limitation to visible and infrared frequencies, which are sensitive only
to the upper parts of the clouds and thus provide an even less direct link to
surface precipitation compared to PMW measurements.

Figure 1 illustrates the key characteristics of the various types of satellite
observations used for the remote sensing of precipitation. Panel (a) presents a
true-color composite from the Advanced Baseline Imager (ABI) aboard GOES-16,
showing Hurricane Laura as an expansive cloud system in the southeastern portion
of the domain. Visible imagery such as this true-color composite can reveal
detailed cloud structures but is limited to daylight hours due to its reliance
on reflected sunlight. Panel (b) displays thermal infrared (IR) imagery with a
wavelength of 11 µm observed from a constellation of geostationary sensors. At
this wavelength, the clear-sky atmosphere exhibits high transmittance, while
clouds are opaque. Consequently, the measured radiances primarily originate from
the Earth’s surface in cloud-free regions and from cloud tops in cloudy regions.
Due to the vertical thermal structure of the atmosphere, the cloud tops appear
as areas of cold brightness temperatures against a relatively warmer background.
Panel (c) shows passive microwave observations at a frequency of 89 GHz,
corresponding to a wavelength of around 0.33 cm. Surface-sensitive passive
microwave observations are characterized by a strong contrast between ocean and
land surfaces. Over the radiatively cold ocean background, Hurricane Laura’s
rainband appears as a region of enhanced brightness temperature due to emission
from liquid hydrometeors. Over land, the surface itself emits strongly at
microwave frequencies, making it difficult to isolate the emission signal from
raindrops. Instead, precipitation is primarily detected through the scattering
signature of large raindrops and ice particles, which reduce the observed
brightness temperatures by attenuating surface emission.

Comparing the three panels to the corresponding radar-based surface
precipitation estimates underscores the strength of passive microwave
observations: while visible and IR imagery primarily depict cloud-top features
that may not correlate with surface rainfall, microwave observations exhibit
higher spatial coherence with surface precipitation. The principal limitation of
passive microwave imagery, however, is its limited spatiotemporal coverage due
to narrower swath widths, longer revisit times, and lower spatial resolution.

\subsection{From Satellite Observations to Precipitation Estimates}

Satellite observations only contain an indirect signal from the near-surface
precipitation and thus require careful processing to produce reliable surface
precipitation estimates. This conversion of satellite measurements into
precipitation estimates is commonly referred to as retrieval. While a
physics-based formulation of precipitation retrieval algorithms is possible,
practical implementations typically require a range of simplifications and
ad-hoc assumptions, such as normally distributed retrieval targets and
measurement errors or close-to linear relationships between atmospheric
variables and satellite observations [4], [5]. Because of these difficulties,
empirical approaches have long been used to produce precipitation estimates from
satellite observations [6], [7]. The rise of deep learning and more recent AI
techniques have led to the development of a range of new machine-learning-based
algorithms that yield promising results [8], [9], [10].


\subsection{The Need for a Unified Benchmark Dataset}

The number of machine-learning-based satellite precipitation retrievals is
growing, however the algorithms described in the literature are difficult to
compare. This is largely because they are typically developed for specific
sensors, regions, time periods, and even resolutions. Given the high
spatiotemporal variability of precipitation, such differences in sampling and
geographic focus significantly influence accuracy metrics, rendering published
results incomparable. Moreover, the performance of empirical retrieval
algorithms is influenced by the volume, quality, and spatiotemporal sampling of
the training and evaluation data, and thus does not solely reflect the intrinsic
qualities of a specific algorithm or model.

This lack of comparability makes it difficult to isolate algorithmic
improvements driven by advancements in machine learning from differences
introduced by the choice of training and evaluation data. To address this
challenge, the International Precipitation Working Group (IPWG) [11], a
permanent Working Group of the Coordination Group for Meteorological Satellites
(CGMS), has established a machine learning working group tasked with developing
a standardized benchmarking dataset for empirical and machine-learning-based
precipitation retrievals. The result of this effort is the Satellite-Based
Estimation and Detection of Rain (SatRain) dataset, an AI-ready, large-scale
benchmark dataset for the development and evaluation of precipitation retrieval
algorithms covering a wide range of observations modalities and multiple climate
zones. Despite its name, the dataset is not limited to rain but includes all
types of precipitation encountered during the training and testing periods.


\subsection{The SatRain Dataset}

The SatRain dataset integrates multi-sensor satellite observations with
gauge-corrected, ground-based radar precipitation estimates. It provides a
large, curated training set over the continental United States (CONUS),
encompassing diverse climate regimes ranging from subtropical humid regions to
arid deserts, mountainous terrain, and temperate to cold continental zones. Data
are available in both a 0.036° regular latitude–longitude grid and the native
sampling of the passive microwave sensors. All input and reference fields are
consistently mapped to these two spatial representations, enabling direct use
for both pixel-based and image-based AI algorithms and ensuring the dataset’s
AI-readiness. The satellite observations span a wide range of sensing modalities
relevant to precipitation remote sensing, including temporally resolved imagery
from geostationary platforms. To support model generalization studies, SatRain
also includes independent test sets from Korea and Austria, covering distinct
climate regimes and incorporating alternative reference measurement techniques.

\section{Methods}

\subsection{Data Sources}


The SatRain dataset integrates four principal data sources: passive microwave
(PMW) sensors, geostationary satellite observations, ancillary environmental
data, reference precipitation estimates from ground-based weather radar
networks.

Since PMW observations over a given region of interest are limited to discrete
overpass times, the dataset is constructed around the overpasses of PMW sensors
to which data from the other sources is added accordingly.

A key challenge in creating the SatRain dataset was reconciling differences in
spatial resolution and sampling across sensors. To balance flexibility with
manageable dataset size, SatRain data is provided in two sampling geometries.
The gridded data is mapped to a shared, regular latitude-longitude grid at
0.036° resolution. This resolution matches the native grid of the global
geostationary IR dataset and remains finer than the effective resolution of
current satellite precipitation estimates. In addition to that, the on-swath
dataset is kept on the native spatial sampling of the PMW base sensor. Because
many precipitation retrieval algorithms have historically been developed on the
native sensor sampling, SatRain retains this format to support these retrieval
designs and cross-comparison against existing operational retrievals such as the
Goddard Profiling Algorithm [9], [12].

The PMW observations used to build the SatRain dataset are sourced from two
different sensors: the GPM Microwave Imager (GMI) [13] aboard the GPM Core
Observatory, and a selection of channels of the Advanced Technology Microwave
Sounder (ATMS) aboard the Suomi National Polar-orbiting Partnership satellite
[14]. The channels included from each sensor are listed in Table 1. The GMI and
ATMS sensors were chosen to represent two ends of the spectrum of PMW
instrumentation used for measuring precipitation. GMI, the flagship PMW sensor
of the GPM constellation, has been designed for precipitation remote sensing and
features optimized spectral coverage and comparably high spatial resolution. In
contrast, ATMS was developed primarily for operational weather forecasting and,
compared to GMI, offers fewer precipitation-sensitive channels and significantly
coarser spatial resolution.

Figure 2 shows GMI and ATMS observations of Hurricane Laura, taken at 12:41 UTC
and 8:41 UTC, respectively, on August 27, 2020. Compared to ATMS, GMI has a
considerably narrower swath and thus remains more limited in its spatial
coverage. However, the GMI sensor offers higher spatial resolution, allowing it
to resolve finer cloud and precipitation structures than ATMS. The two sensors
also differ in their scanning techniques: GMI is a conical scanner, meaning its
antenna beam sweeps out a cone with a nearly constant Earth-incidence angle. As
a result, its polarization and field of view characteristics remain consistent
across the swath. ATMS, in contrast, uses a cross-track scanning technique. This
causes the antenna footprint to increase in size toward the edges of the swath,
reducing spatial resolution. Additionally, variations in ray path length and
polarization angle introduce systematic changes in observation across the scan.

\subsubsection{Geostationary Observations}

A principal limitation of passive microwave observations is their restricted
spatial and temporal coverage, as they are confined to the discrete overpass
times of low-Earth orbiting satellites. In contrast, visible (Vis) and infrared
sensors can achieve higher spatial resolutions due to the shorter wavelength of
the measured radiation allowing them to be deployed on geostationary platforms.
This allows for continuous monitoring of the hemisphere below. As a result,
geostationary observations are essential for real-time and continuous
precipitation monitoring. From a machine-learning perspective, these
observations provide important spatiotemporal information on clouds and
environmental context for developing better precipitation estimation models.

Figure 3 shows observations from the 16 spectral channels of the Advanced
Baseline Imager aboard GOES-16 during the landfall of Hurricane Laura. The
native spatial resolution of ABI channels ranges from 500 meters to 2 kilometers
at the sub-satellite point on the equator. Although this resolution degrades
over the Continental United States due to increasing viewing angles, it remains
substantially higher than that of PMW sensors. The first six channels are
visible and near-infrared bands that measure reflected solar radiation and are
only available during daylight hours. The remaining ten channels operate in the
thermal infrared and provide data continuously, both day and night. Among the
thermal IR bands, the main distinguishing feature is their sensitivity to
atmospheric water vapor. For instance, channels centered at 6.2, 6.9, and 7.3 µm
are more sensitive to upper-tropospheric moisture and thus have shallower
penetration depths. This sensitivity to water vapor provides contextual
information on the moisture content of the air.

Latest-generation geostationary platforms such as the GOES R-series [15],
Himawari 8 and 9 [16], and MeteoSat Third Generation [17] provide observations
at temporal resolution of 10 minutes allowing them to closely track the
evolution of precipitation systems. In order to allow users to explore the
temporal information content in time-resolved geostationary observations, the
SatRain dataset includes observations from multiple time steps around the
overpass of the PMW sensor.

In addition to the multi-channel visible and infrared observations from the
latest generation of geostationary sensors, the SatRain dataset also integrates
0.036-degree gridded thermal IR observations from the 11 um infrared window
sourced from the Climate Prediction Center (CPC) global gridded geostationary IR
dataset. Since these observations are available almost continuously from 1998,
they play an important role for generating long-term precipitation records and
are included as an independent input data source in the SatRain dataset.

\subsubsection{Ancillary Data}

Because the relationship between satellite observations and surface precipitation is often under-constrained, it is common to augment satellite observations with complementary environmental information, so-called ancillary data, to improve the accuracy of the precipitation estimates. Typical examples include the surface type, atmospheric and surface temperatures, humidity, and elevation. The SatRain dataset includes several static and dynamic ancillary variables. Dynamic ancillary data describing the state of the atmosphere and the surface are derived from the ERA5 [18] dataset. In addition to that, the ancillary data also contains an 18-class surface classification that has been developed for the Goddard Profiling Algorithm (GPROF, ) precipitation retrieval and combines microwave-based surface-type information with snow- and sea-ice coverage data from the Autosnow product [19]. In terms of static variables, SatRain provides the surface elevation sourced from the NOAA Global Land One-kilometer Base Elevation (GLOBE) digital elevation model [20].

\subsubsection{Precipitation Reference}

The precipitation reference used in the SatRain dataset are derived from gauge-corrected ground-based precipitation radar measurements from NOAAs gauge-corrected Multi-Radar Multi-Sensor (MRMS) [21] product. MRMS is based on radar observations from Nexrad, the most extensive network of precipitation radars in the world comprising around 160 polarimetric S-band radars. The radar-derived estimates of liquid precipitation are corrected using hourly gauge-correction factors thus enforcing consistency between instantaneous estimates and direct measurements of hourly accumulations from gauge stations. While a certain level of residual uncertainty in these estimates cannot be eliminated, they are generally considered to be the best currently available estimates of surface precipitation with near-complete coverage over CONUS. In addition to reference surface precipitation rates, the SatRain dataset contains a radar quality index and the gauge correction factor, allowing the user to customize the quality requirements for the radar estimates used during both training and evaluation. Furthermore, the dataset contains precipitation-type masks identifying convective and stratiform rain, snow, and hail as provided by the MRMS data. Examples of these fields are shown in Fig. 5.

\subsubsection{Independent Test Data}

Since the SatRain training data is derived from four years (2019 - 2021) of ground-based radar measurements over CONUS, evaluating ML retrievals using this same data may lead to overfitting to specific weather patterns or measurement errors in the MRMS-based precipitation reference. To reduce the risk of overfitting on the weather patterns from the training period, the test data is derived from the year 2022. Moreover, the SatRain dataset includes two additional test datasets derived from different geographical regions and measurement systems.

The first geographically independent evaluation dataset consists of gauge-corrected, ground-based radar rainfall estimates over South Korea, generated using radar compositing methods optimized for the Korean domain [22]. The Korea-specific merging technique uses radial basis function interpolation to combine radar and rain gauge observations resulting in a high-resolution rainfall product in both space and time [23]. Although the measurement approach is similar to that used in the MRMS reference data over CONUS, the processing of raw data into precipitation estimates differs. The dataset also represents a distinct climate regime from the training data, offering a valuable test of the precipitation retrieval model’s ability to generalize beyond CONUS conditions. Studies such as [24] and [25]
 showed that cloud characteristics affecting both PMW radiometers and IR observations differ substantially between Korea and CONUS and can cause significant biases in algorithms applied to these two regions. The independent testing data from Korea thus provides an opportunity to assess whether retrieval improvements are achieved at the cost of global generalizability.

The second independent evaluation dataset is derived from gauge measurements from the WegenerNet [26] gauge network around the Feldbach region in Austria. While IPWG focuses primarily on gauge-corrected radar data for validation, WegenerNet is unique in that its gauge density is sufficiently high that the addition of radar data would not modify any of the gauge accumulations. In addition, it offers validation over a mountainous regime that radar/gauge networks still struggle with. For the comparison against satellite-based precipitation estimates the measurements from the ground stations were aggregated to the 0.036-degree grid using binning. As can be seen in Panel (b) of Fig. 6, the gauge density is sufficient to cover 24 grid cells with at least two gauges per cell. The mapping of the gridded measurements to the on-swath geometry was performed using nearest-neighbor interpolation.


\subsection{Dataset Generation}

In contrast to observations from geostationary sensors and ground-based reference data, passive microwave observations from GMI and ATMS are available only at discrete overpass times. The SatRain dataset is generated by extracting observations from all available overpasses of the GMI and ATMS sensor during the time period January 2018 to January 2023 and adding the corresponding geostationary observations, ancillary data, and ground-based reference precipitation estimates. The resulting collocation scenes are then used to extract fixed-size training scenes that can be used to train both pixel-based as well as scene-based machine-learning-based precipitation estimation and detection models.

The training, validation, and testing splits of the SatRain dataset are chosen so that they are always either temporally or spatially independent. The training and validation data are extracted from the years 2019 up and including 2021. Data from the five first days of every month are assigned to the validation set and data from the remaining days to the training data set. The CONUS testing data is derived from the year 2022. The independent testing data over Austria uses observations from the years 2021 and 2022. The independent testing data over Korea uses observations from October 2022 to October 2023.

\subsubsection{Generation of Collocation Scenes}

The first step of the generation of the SatRain dataset is the generation of
collocation scenes for every overpass of the PMW base sensor, i.e., GMI or ATMS,
over the targeted domain. The resulting collocation scene contains all retrieval
input data, i.e. satellite observations and ancillary data, combined with the
coincident reference precipitation on a common spatial grid. Two collocation
scenes are extracted for every overpass of the base sensors: A on-swath scene,
which uses the native sampling of the PMW observations, and a gridded scene,
which contains all data regridded to a regular latitude-longitude grid with a
resolution of 0.036 degree.

The collocation process, as illustrated in Fig. 6, starts out with the PMW
observation from an overpass of GMI or ATMS over the domain containing the
reference data. Corresponding ground-based reference data and ancillary data are
extracted to cover the time range of the overpass and interpolated to the
observation time of each scan-line of the PMW observations. The MRMS data, which
have a native resolution of 0.01 degree, are reduced in resolution to match
0.036-degree grid by smoothing using a Gaussian filter with a full width at half
maximum of 0.036 degrees and interpolated linearly to the target grid. The
mapping of the reference precipitation estimates to the on-swath geometry is
performed by nearest-neighbor interpolation. Similarly, the PMW observations are
mapped to the regular latitude-longitude grid by nearest neighbor interpolation.

The global gridded IR geostationary observations are extracted over a time
window of 8 hours centered on the median overpass time at a temporal resolution
of 30 min. Multi-channel Vis and IR observations from the GOES ABI are extracted
over a time window of 1 hour and a temporal resolution of 10 minutes. The Vis/IR
observations are mapped to the gridded and on-swath geometries using
nearest-neighbor interpolation.

Because GOES ABI observations are limited to CONUS, the Vis/IR data for the two
additional test domains are taken from the corresponding geostationary sensors
covering those regions. For the Korea domain, Vis/IR observations are drawn from
the AHI instrument aboard the Himawari-8 and -9 satellites. For the Austria
domain, the data are sourced from the SEVIRI instrument on the Meteosat Second
Generation platform.

\subsubsection{Training Scene Extraction}


To generate AI-ready training data from the collocation dataset, fixed-size
training scenes are extracted from the previously created collocation dataset.
Two separate sets of scenes are extracted for the gridded and on-swath
observation geometries. The scenes are extracted randomly, allowing an overlap
of up to 50\% between neighboring scenes, requiring 75% of the pixels to contain
valid observations and reference data. The scene size for the gridded data is
256 x 256 and 64 x 64 for the on-swath data. The smaller size of 64 x 64 used
for the on-swath data is due to the limited width of the swaths of the PMW
observations, which is 96 for ATMS and 221 for GMI. The scene extraction process
is illustrated in Fig. 7.

\section{Data Records}

\subsection{Dataset Organization}

The SatRain dataset consists of two independent subsets corresponding to collocations scenes extracted from GMI overpasses and ATMS observations, respectively. The underlying PMW sensor of each subset is referred to as the base sensor of the corresponding subset. For each base sensor, the data is split into training, validation, and testing splits following machine-learning best practices. The training and validation splits share the same data format. The validation split contains samples from the first five days of each month and the training split the remaining data. While the validation set is generally intended for monitoring training progress, this is left to the user, who may also merge the two splits and use both for training.
The data in the testing split consists of the complete collocation scenes instead of the fixed-size scenes used for the training and validation splits. The test scenes retain the original observation structure in order to avoid distortion of the sampling statistics due to the randomized extraction of fixed size scenes. Additionally, retaining the original observation structure also makes it easier to assess already existing precipitation products using the test data. Finally, while the testing data is available in the on-swath geometries for both base sensors, the retrieval accuracies should be assessed against the gridded reference data. This is relevant for sensors like ATMS for which evaluation using the on-swath geometry would underestimate the effect of the reduced resolution at the swath edges.

The training and validation splits are split once more into subsets of different sizes: ‘xs’, ‘s’, ‘m’, ‘l’, and ‘xl’. This additional splitting of the data was performed to support users and compute systems with limited storage resources, while also providing a dataset large enough to train modern deep learning models.

The testing split is split into three spatial domains: ‘conus’, ‘korea’, ‘austria’, corresponding to the underlying reference data: MRMS over CONUS, ground-based radar data over Korea, and in-situ measurements from the WegenerNet stations in Austria.

\subsection{Data Files}

The various input and target data for each training scene are stored in separate
NetCDF4 files. This modular organization allows users to download just the data
they intend to use, for example, only the small (‘s’) subset of the GMI and
target data to train and evaluate a retrieval using only passive microwave
observations. Each file is identified using an individual prefix (‘gmi’, ‘atms’,
‘geo’, ‘geo\_ir’, ‘ancillary’, ‘target’) following a time stamp containing the
median observation time. Input and target files corresponding to a specific
training, validation, or testing sample can thus be identified using this
timestamp.

Due to the large size of the time-resolved geostationary observations, the
geostationary observations are split into files containing only the observations
closest to the reference precipitation estimates and files containing
observations from multiple observations times. The multi-timestep observations
are stored in separate files with the suffix ‘\_t’, i.e., ‘geo\_t’ and ‘geo\_ir\_t’.

\subsection{Passive Microwave Observations}

The PMW observations in the SatRain dataset consist of observations from the GMI
sensor and the ATMS sensor on the NOAA-20 satellite. They are stored in files
labeled ‘gmi\_\<timestamp>.nc’ for the subset using GMI as base sensor and,
correspondingly, atms\_\<timestamp>.nc’ for the subset using ATMS as base sensor.
Each file contains the brightness temperatures in Kelvin for each frequency
channel, the corresponding earth-incidence angles, and the observation time
corresponding to each scan line.

\subsection{Geostationary IR observations}

The single-channel CPC gridded IR observations are stored in files named
‘geo\_ir\_<timestamp>.nc’ and contain IR window-channel observations from
wavelengths around 11 um. The observed brightness temperatures in Kelvin are
stored in a variable called observations. The single-timestep files contain the
GEO IR observations closest to the measurement time of the reference
precipitation estimates. The temporally-resolved GEO IR observations contain 16
half-hourly observations centered on the median observation time and are stored
in files named ‘geo\_ir\_t\_<timestamp>.nc’.

\subsection{Geostationary IR/VIS observations}

The multi-channel VIS/IR observations are stored in files named
‘geo\_<timestamp>.nc’. These files contain the observations closest in time to
the measurement time of the reference precipitation estimates. The visible
channels are stored as reflectances while thermal IR channels are stored using
brightness temperatures. The geostationary data currently does not include
viewing angles so the user will have to add them manually.

The multi-timestep files (geo\_t\_<timestamp>.nc’) contain observations from four
10-minute time steps prior to the collocation median and three 10-minute time
steps after the collocation median time. Since GOES observations are only
available over CONUS, the geostationary observations for the test data from the
Korea and Austria domains are derived from Himawari-9 and Meteosat satellites.
Since the corresponding sensors have different channels, the observations differ
in their spectral coverage and users must account for that in their algorithm
design.

\subsection{Ancillary Data}

The ancillary data is stored in separate files named ‘ancillary\_<timestamp>.nc’.
Each file contains the ancillary variables listed in table 2.


\subsection{Target Data}

The target data corresponding to the retrieval input data is stored in a file
target\_<timestamp>.nc, where the timestamp matches that of the input data. The
reference data files contain the surface precipitation in a variable called
‘surface\_precip’. Additionally, the reference data files derived over CONUS
contain several quality indicator variables that can be used to filter the
reference data samples used for training and validation. The primary quality
indicator is the radar quality index, which provides an estimate of the quality
of the surface precipitation estimates based on the radar beam height, beam
blockage, and the height of the freezing level. Moreover, the files contain the
gauge correction factor that was applied to correct the radar-only precipitation
estimates. Finally, the files also contain the fraction of valid, snowing, and
hailing 0.01-degree-resolution pixels within the downsampled 0.036 degree grid
box of the SatRain dataset. Since the radar data over Korea and the station data
over Austria do not provide this additional information, these auxiliary fields
are not provided by the target data files from the ‘Korea’ and ‘Austria’
domains.


\section{Technical Validation}

To demonstrate the practical value of the dataset as well as its technical
correctness, we have used the CONUS-based training data to train three different
precipitation retrievals using the three principal input-observations types,
i.e., multi-channel geostationary observations from the GOES ABI (GEO),
single-channel IR observations from the CPCIR dataset (GEO IR), and PMW
observations from GMI (GMI), and applied them to retrieve precipitation during
the landfall of Typhoon Khanun over Korea. The retrievals use satellite
observations only and do not make use of the ancillary data provided by the
SatRain dataset. The retrievals are implemented using a relatively shallow,
convolutional neural network with 10 million parameters. The resulting reference
and retrieved precipitation fields are compared to the reference ground-based
radar estimates, and two baselines from the ERA-5 reanalysis dataset and the
GPROF precipitation retrieval in Fig. 8.

The results demonstrate that all retrievals capture the main precipitation
structures of Typhoon Khanun, but with clear differences in accuracy reflecting
the information content of their input observations [8], [27]. The PMW-based
retrieval performs best, reproducing much of the fine-scale structure evident in
the reference estimates. The multi-channel geostationary retrieval captures the
primary precipitation bands but misses finer details resolved by PMW. In
contrast, the single-channel IR retrieval shows the weakest performance, with
limited structural detail and correspondingly lower linear correlation and
higher mean-squared error.


 To demonstrate the generalizability of the results shown in Fig. 8, we assess
 the retrievals trained on the CONUS dataset on the three testing datasets
 (CONUS, Austria, and Korea). The results of the retrievals trained on the
 SatRain dataset are displayed together with the resulting metrics of the ERA5
 and GPROF V7 baseline results in Fig. 8. The three SatRain retrievals exhibit
 large biases over Austria and Korea. This is likely a result of the retrievals
 being trained using only data from the CONUS. The impact of regional
 precipitation characteristics on regional precipitation accumulations has been
 demonstrated in studies such as . Moreover, these large biases also affect
 operational precipitation products such as GPROF. For the other error metrics,
 however, the relative results from the evaluation over the Austria and Korea
 domains are consistent with the retrieval results obtained over CONUS. Although
 the absolute values of the metrics differ between the three test datasets, the
 ranking remains the same indicating that benchmark that the corresponding
 accuracy metrics generalize to independent spatial domains, measurement
 techniques, and temporal periods.

A notable exception to this is the accuracy of the geostationary Vis/IR
retrieval over Austria, which is lower than over the other domains. This is due
to the channels of the SEVIRI sensors available over this domain differing from
the channels of the GOES ABI. Although we have selected a channel subset
matching those of the ABI the accuracy still degraded indicating that slight
differences in central wavelength, bandwidth, and calibration of the channels
impacts the retrieval negatively.

Beyond evaluating across observation modalities, we also assessed how well the
dataset can quantify the skill of different machine-learning approaches. Using
GMI observations from SatRain, we trained four retrieval models based on
distinct techniques: Random Forests, XGBoost, a multi-layer perceptron (MLP),
and a convolutional neural network (CNN). Figure 9 compares their performance.
The results reveal a consistent ranking of methods across the domain, with
neural network–based approaches yielding the most accurate retrievals.
Importantly, these accuracy gains extend beyond the regions used for training,
underscoring their ability of the results to generalize outside the training
domain.

Taken together, these findings highlight the value of SatRain as a benchmark
dataset: it not only supports the development of precipitation retrievals but
also enables systematic evaluation and comparison of different machine-learning
techniques. We anticipate it will serve as a key resource for researchers aiming
to develop novel retrieval methods.

\section{Usage Notes}

Different scientific and societal applications may require precipitation
retrievals to emphasize specific characteristics, such as convective or frozen
precipitation. While no single benchmark can serve every possible use case, the
SatRain dataset defines five distinct evaluation tasks designed to test the
ability of machine-learning algorithms to reproduce key aspects of precipitation
events. These tasks include: (1) precipitation rate estimation, (2)
probabilistic detection of precipitation, (3) deterministic detection of
precipitation, (4) probabilistic detection of heavy precipitation, and (5)
deterministic detection of heavy precipitation. We adopt thresholds of 0.1 mm/h
and 10 mm/h to define precipitation and heavy precipitation, respectively.
Figure 10 illustrates example results for each of these tasks using the GMI
retrieval presented in the previous section.

\subsection{Evaluation Protocol}

To ensure fair comparison of precipitation retrievals trained on the SatRain
dataset, it is essential that models are evaluated using a consistent set of
criteria. We therefore propose a standardized evaluation protocol for
benchmarking machine-learning retrievals on the SatRain dataset. Users may
choose to evaluate their models on all or a subset of the defined tasks. For
each task, the recommended accuracy metrics are listed in Table 4:

A reference implementation of this protocol is available in the ipwgml package.
By default, the evaluation compares all retrieval outputs against the reference
precipitation estimates on the 0.036° regular latitude-longitude grid, ensuring
results are independent of the retrieval’s native coordinate system. For the
CONUS domain, evaluations should be restricted to regions with a radar-quality
index of at least 0.5 and should include both snow and hail.


\subsection{The ipwgml package}

To simplify access to SatRain and accelerate adoption by the community, we have
developed the ipwgml Python package. The package provides utilities to automate
dataset download and management, ensuring that users can begin working with the
dataset without the overhead of manual data handling. Comprehensive
documentation is available at ipwgml.readthedocs.org including installation
instructions, usage examples, and tutorials.

Beyond data access, the package implements the standardized evaluation protocol
defined for SatRain. This functionality allows users to benchmark retrieval
models trained directly on SatRain, as well as to assess independently developed
retrievals against the same criteria. In doing so, the ipwgml package ensures
that evaluation results are reproducible, consistent, and directly comparable
across studies.

\subsection{Limitations}

The SatRain dataset is constructed from high-quality input datasets using
state-of-the-art techniques designed to reduce uncertainties in both the
satellite observations and the precipitation reference. Despite these efforts,
residual uncertainties and measurement errors remain in both components.

On the input side, obviously corrupted satellite imagery is flagged and removed
from the satellite-observation data used to construct the SatRain dataset.
However, more subtle issues such as undetected artifacts or gradual changes in
sensor characteristics may persist and affect the data. These represent
practical challenges that any precipitation retrieval must contend with
warranting their inclusion in the SatRain dataset.

The precipitation reference data are also subject to significant uncertainties.
While gauge-corrected, ground-based radar composites are widely regarded as the
most reliable spatially continuous precipitation estimates currently available,
they are not error-free. Beam overshooting, uncertainties in microphysical
processes, and the particular difficulty of estimating snowfall introduce
systematic biases. Snowfall poses a notable challenge: most gauges do not
accurately measure snow, and MRMS does not apply gauge correction to snowfall
estimates. Consequently, snowfall included in SatRain training and testing data
should be regarded as highly uncertain. For the WegenerNet test data, only
heated gauges are used under likely snowfall conditions, but limitations remain.

To reduce the impact of these uncertainties on retrieval evaluation, SatRain
includes independent test datasets from geographically distinct domains that
rely on entirely independent measurement systems. Performance gains that
transfer to these independent datasets are more likely to reflect genuine
improvements in retrieval capability, rather than overfitting to the same
reference data used in training.

It is also important to note that the primary purpose of SatRain is to serve as
a benchmark for evaluating and comparing retrieval algorithms, rather than as a
basis for developing globally accurate precipitation retrievals. The SatRain
dataset is limited to training data over the CONUS and is therefore not designed
for the development of global precipitation retrievals. Algorithms trained on
SatRain will learn to capture regional precipitation characteristics specific to
North America, which may result in substantial biases or retrieval errors when
applied to other parts of the world.

\subsection{Future Directions}

The SatRain dataset represents the first AI-ready benchmark for satellite-based
precipitation estimation and detection, marking an important step toward
facilitating the operational adoption of machine-learning-based retrievals. It
brings together a comprehensive set of satellite observation types alongside
high-quality precipitation reference data derived according to best practices
agreed upon by the international community. By providing a common reference
point, SatRain enables systematic assessment of machine-learning advances in
precipitation retrieval, improving the comparability, reproducibility, and
eventual uptake of retrieval techniques reported in the scientific literature.

Looking ahead, SatRain also opens opportunities for advancing retrieval science
beyond single-sensor approaches. Its multi-sensor, multi-timestep design
provides a strong foundation for developing next-generation algorithms that fuse
observations across sensors and time steps to better capture precipitation
structure and evolution. Furthermore, the dataset is ideally suited for tackling
broader challenges in global precipitation retrieval, including the development
of sensor-agnostic retrievals and the mitigation of regional biases. By
providing a common starting point to address these challenges, SatRain can help
guide the community toward more robust, transferable, and operationally relevant
precipitation retrieval systems.


\bibliographystyle{plainnat}
\bibliography{refs}
\end{document}
